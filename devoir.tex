\documentclass{article}
\usepackage[utf8]{inputenc}
\usepackage[T1]{fontenc}
\usepackage{color}
\usepackage{listings}
\usepackage{graphicx}
\usepackage{fancyhdr}
\pagestyle{fancy}
\chead{en-tête centre}
\cfoot{pied-de-page centre}


\title{devoir}
\author{DE SOUZA Brener}
\date{\today}

\begin{document}

\maketitle
\pagebreak

\tableofcontents
\listoffigures
\listoftables


\section{Première section}
\subsection{première sous section}

\section{Deuxième section}
\subsection{première sous section}

Un algorithme: \\

\begin{lstlisting}[language=c]
#include <stdio.h>
int main() {
   printf("Hello, World!");
   return 0;
\end{lstlisting}

deux formules mathématiques: 

\begin{equation} \\
    x=\frac{-b+\sqrt{b^{2}\times4ac}}{2a}
\end{equation}

\begin{equation} \\
    f(x) = \sqrt{x^{2}+1}
\end{equation}

Une Liste de langages:
\begin{itemize}
    \item C
    \item C++
    \item Python
    \item Java
\end{itemize}

Une référence: \\

"L'espionne de Tanger (2009) \cite{maria} est un roman de l'écrivaine espagnole María Dueñas. C'est un récit très bien conçu sur la vie trépidante de Sira Quiroga, une jeune couturière qui a quitté Madrid quelques mois avant la guerre civile. En attendant, pour le lecteur, l'approche de l'auteur d'un contexte historique critique en Espagne et en Europe est révélatrice."

\begin{thebibliography}{1}
\bibitem{maria} María Dueñas. L'espionne de Tanger, 2009.

\end{thebibliography}

Une figure Overleaf\footnote{au dos de la page}:
\begin{figure}[h]
    \centering
    \includegraphics{devoirs/overleaflogop2.png}
    \caption{Caption}
    \label{fig:my_label}
\end{figure}

Un tableau:
\begin{table}[h]
    \centering
    \begin{tabular}{l|c||r}
    left & center & right \\ \hline
        1 & 2 & 3 \\
        4 & 5 & 6 \\
        7 & 8 & 9 \\
    \end{tabular}
    \caption{Des Numéros}
    \label{tab:un_label}
\end{table}  

\end{document}